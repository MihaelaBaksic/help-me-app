\chapter{Arhitektura i dizajn sustava}
			
			
			Odabir odgovarajuće arhitekture je važna odluka koja utječe na cjelokupnu funkcionalnost sustava. Budući da je cilj aplikacije široka dostupnost, odabrali smo web aplikaciju. Web aplikacija funkcionira neovisno o platformi, što oslobađa razvojni tim višeplatformnog razvoja.\\
			Korisnik aplikaciji pristupa pomoću web preglednika. Web preglednik je program koji omogućava korisniku pregled i prikaz web aplikacije na uređaju. Kako bi se omogućila komunikacija klijenta(korisnika) s aplikacijom koristi se web poslužitelj koji koristi HTTP protokol. Aplikacija komunicira sa backend-om preko REST API-ja. Backend dohvaća sve potrebne podatke iz baze podataka(više o njoj u sljedećoj sekciji) nakon čega preko poslužitelja i preglednika prikazuje te podatke korisniku u obliku HTML dokumenta.\\
			Aplikacija je pisana u programskom jeziku Java i programskom okruženju Spring Boot, te React-u(JavaScript biblioteka za izgradnju korisničkih sučelja). Za arhitekturu web aplikacije odabrali smo MVC(engl. Model-View-Controller) obrazac. Budući da MVC odvaja pojedine dijelove aplikacije ovisno o njihovoj namjeni, omogućava se paralelan razvoj te jednostavna nadogradnja.\\
			Naša aplikacija koristi MVC na sljedeći način:
			
			\begin{itemize}
				\item \textbf{Model} – centralna komponenta obrasca. Sadržava pravila, logiku te bazu podataka. Kako bi implementirali ovu komponentu MVC obrasca koristiti ćemo servise Spring Boot-a, Java objekte za poslovnu logiku i PostgreSQL bazu podataka. 
				\item \textbf{View} – vizualni dio aplikacije s kojim korisnik interaktira. Kako je naša aplikacija web aplikacija, View će biti korisničko sučelje ostvareno pomoću React-a, HTML-a i CSS-a. 
				\item \textbf{Controller} –  prima i obrađuje svaku vrstu zahtjeva korisnika, te ih šalje Modelu ili Viewu. Budući da koristimo Spring Boot radni ovir, konvencija nalaže da se ovaj dio ostvari pomoću REST konrolera. Tu konvenciju poštuje i ova aplikacija. Svi se zahtjevi prosljeđuju koristeći JSON format.
			\end{itemize}
			\begin{figure}
				\includegraphics{Arhitektura_sustava}
				\caption{Arhitektura sustava}
			\end{figure}
			
			
			
			\section{Baza podataka}
			
			Sve je podatke potrebno negdje spremiti kako bi se mogli dinamički dohvaćati. Za ovo nam služi baza podataka, koju također smatramo dijelom MVC obrasca. Naša aplikacija u pozadini koristi, kao relacijsku bazu podataka, PostgreSQL.
			Sve relacije su dovedene u 3. normalnu formu.\\
			U bazi podataka nalaze se sljedeći entiteti:
			\begin{itemize}
				\item Korisnik
				\item Adresa
				\item Ocjena
				\item Zahtjev
				\item Potencijalni
			\end{itemize}
			
			\subsection{Opis tablica}
			
			
			\textbf{Korisnik} Ovaj entitet modelira jednog korisnika aplikacije.\\
			Sadrži atribute: korisnikID, ime, prezime, e-posta, lozinka, korisnickoIme, jeAdmin, telefon, slika, status, vrijemeBlokiranja i
			adresaID koji predstavlja strani ključ na entitet Adresa.
			
			\begin{tabularx} {\textwidth} {|p{3.5cm}|p{2cm}|X|}
				
				\hline
				\multicolumn{3}{|c|}{\textbf{Korisnik}} \\
				\hline

				
				\cellcolor{LightGreen}korisnikID & INT	&   jedinstveni identifikator svakog korisnika 	\\ \hline
				ime	& VARCHAR &   ime korisnika	\\ \hline 
				prezime & VARCHAR &  prezime korisnika \\ \hline 
				e-posta & VARCHAR	&  	e-mail adresa korisnika	\\ \hline 
				lozinka & VARCHAR	&  	hash lozinke	\\ \hline
				korisnickoIme & VARCHAR	&  	korisnicko ime	\\ \hline
				jeAdmin & BOOLEAN	&  	oznaka je li korisnik administrator	\\ \hline
				telefon & VARCHAR	&  	broj mobitela korisnika	\\ \hline
				slika & BOOLEAN	&  	oznaka je li korisnik ima sliku profila	\\ \hline
				status & VARCHAR	&  	oznaka statusa korisničkog računa	\\ \hline
				vrijemeBlokiranja & DATETIME	&  vrijeme blokiranja korisnika		\\ \hline
				\cellcolor{LightBlue}adresaId & VARCHAR	&  	adresa prebivališta korisnika	\\ \hline
				
				
				
			\end{tabularx} 
			
			\bigskip
			\bigskip
			\textbf{Adresa} Ovaj entitet modelira adresu prebivališta pojedinog korisnika aplikacije.
			Sadrži sljedeće atribute: adresaID, ulica, broj, pbr, imeMjesto.
			\bigskip
			
			
			\begin{tabularx} {\textwidth} {|p{3.5cm}|p{2cm}|X|}

				\hline
				\multicolumn{3}{|c|}{\textbf{Adresa}} \\
				\hline
				
				\cellcolor{LightGreen}adresaID & INT	& jedinstveni adrese korisnika	\\ \hline
				ulica	& VARCHAR &  naziv ulice 	\\ \hline 
				broj & INT & kućanski broj  \\ \hline 
				pbr & VARCHAR	&  	poštanski broj mjesta	\\ \hline 
				imeMjesto & VARCHAR	&  naziv mjesta		\\ \hline
				
				
				
			\end{tabularx}
			
			\bigskip
			\bigskip
			\textbf{Ocjena} Ovaj entitet predstavlja ocjenu koju jedan korisnik daje drugome. Sadrži atribute: ocjenaID, komentar, ocjena, korisnikID, zahtjevID, primakorisnikID. Ovaj entite sadrži tri strana ključa, a to su: korisnikID(predstavlja korisnika koji ocjenjiva), primakorisnikID(predstavlja korisnika kojeg se ocjenjiva) i zahtjevID(predstavlja zahtjev koji se izvršava).
			\bigskip
			
			\begin{tabularx} {\textwidth} {|p{3.5cm}|p{2cm}|X|}
				
				\hline
				\multicolumn{3}{|c|}{\textbf{Ocjena}} \\
				\hline
				
				\cellcolor{LightGreen}ocjenaID & INT	& jedinstveni identifikator svake ocjene	\\ \hline
				komentar	& VARCHAR &  komentra kojeg korisnik ostavlja uz ocjenu 	\\ \hline 
				ocjena & INT & ocjena koju korisnik dodjeljuje  \\ \hline 
				\cellcolor{LightBlue} korisnikID	& INT &  korisnik koje ocjenjiva 	\\ \hline 
				\cellcolor{LightBlue} zahtjevID	& INT &  zahtjev koji se izvršava 	\\ \hline 
				\cellcolor{LightBlue} primakorisnikID	& INT &  korsnik kojeg se ocjenjuje 	\\ \hline 
				
			\end{tabularx}
			
			\bigskip
			\bigskip
			\textbf{Zahtjev} Ovaj entitet predstavlja jedan zahtjev kojeg korisnik aplikacije zadaje ili izvršava. Sadrži atribute:zahtjevID, opis, datumVrPocetka, trajanje, status, adresaID, korisnikID, autorskikorisnikID. Kao i entitet ocjena i ovaj entitet sadrži tri strana ključa: adresaID(predstavlja adresu korisnika koji je zadao zahtjev, nije obavezno kako bi se kreirao zahtjev), korisnikID(predstavlja izvršitelja zahtjeva) i autorskikorisnikID(predstavlja samog kreatora zahtjeva).
			\bigskip
			
			\begin{tabularx} {\textwidth} {|p{3.5cm}|p{2cm}|X|}
				
				\hline
				\multicolumn{3}{|c|}{\textbf{Zahtjev}} \\
				\hline
				
				\cellcolor{LightGreen}zahtjevID & INT	& jedinstveni identifikator svakog zahtjeva	\\ \hline
				opis	& VARCHAR &  kratki opis zahtjeva 	\\ \hline 
				datumVrPocetka & DATETIME &  trenutak postavljanja zahtjeva na aplikaciju \\ \hline 
				trajanje & TIME	&  vremenski period u kojem se zahtjev može izvršiti		\\ \hline 
				status & VARCHAR & status zahtjeva  \\ \hline
				\cellcolor{LightBlue} adresaID	& INT &  adresa autora zahtjeva 	\\ \hline 
				\cellcolor{LightBlue} korisnikID	& INT &  izvršitelj zahtjeva 	\\ \hline
				\cellcolor{LightBlue} autorskikorisnikID	& INT &   autor zahtjeva	\\ \hline
				
				
			\end{tabularx}
			
			\bigskip
			\bigskip
			\textbf{Potencijalni} Ovaj entitet predstavlja sve potencijalne izvršitelje jednog zahtjeva. Sadrži atribute: zahtjevID, korisnikID. Oba atributa su strani ključevi. Prvi predstavlja zahtjev kojeg korisnik želi izvršit, drugi predstavlja korisnika koji želi izvršit zahtjev.
			\bigskip
			
			\begin{tabularx}{\textwidth} {|p{2cm}|p{2cm}|X|}
				\hline
				\multicolumn{3}{|c|}{\textbf{Potencijalni}} \\
				\hline
				\cellcolor{LightGreen} zahtjevID & INT & zahtjev kojeg korisnik želi izvršiti \\
				\hline
				\cellcolor{LightGreen} korisnikID & INT & potencijalni izvršitelj zahtjeva \\
				\hline
			\end{tabularx}
			
			
			
			\newpage
			\subsection{Dijagram baze podataka}
			Podcrtani elementi su ključevi, elementi koji imaju (O) nisu obavezni za unos u bazu podataka, elementi koji imaju (FK) su strani ključevi i elementi s oznakom (U) moraju biti jedinstveni.
			
			\begin{figure}[h]
				\includegraphics[height=0.53\textheight]{Relacijski_model}
				\caption{Relacijski model baze podataka}
			\end{figure}
			
			\eject

			
			
		\section{Dijagram razreda}
		
			\textit{Potrebno je priložiti dijagram razreda s pripadajućim opisom. Zbog preglednosti je moguće dijagram razlomiti na više njih, ali moraju biti grupirani prema sličnim razinama apstrakcije i srodnim funkcionalnostima.}\\
			
			\textbf{\textit{dio 1. revizije}}\\
			
			\textit{Prilikom prve predaje projekta, potrebno je priložiti potpuno razrađen dijagram razreda vezan uz \textbf{generičku funkcionalnost} sustava. Ostale funkcionalnosti trebaju biti idejno razrađene u dijagramu sa sljedećim komponentama: nazivi razreda, nazivi metoda i vrste pristupa metodama (npr. javni, zaštićeni), nazivi atributa razreda, veze i odnosi između razreda.}\\
			
			\textbf{\textit{dio 2. revizije}}\\			
			
			\textit{Prilikom druge predaje projekta dijagram razreda i opisi moraju odgovarati stvarnom stanju implementacije}
			
			
			
			\eject
		
		\section{Dijagram stanja}
			
			
			\textbf{\textit{dio 2. revizije}}\\
			
			\textit{Potrebno je priložiti dijagram stanja i opisati ga. Dovoljan je jedan dijagram stanja koji prikazuje \textbf{značajan dio funkcionalnosti} sustava. Na primjer, stanja korisničkog sučelja i tijek korištenja neke ključne funkcionalnosti jesu značajan dio sustava, a registracija i prijava nisu. }
			
			
			\eject 
		
		\section{Dijagram aktivnosti}
			
			\textbf{\textit{dio 2. revizije}}\\
			
			 \textit{Potrebno je priložiti dijagram aktivnosti s pripadajućim opisom. Dijagram aktivnosti treba prikazivati značajan dio sustava.}
			
			\eject
		\section{Dijagram komponenti}
		
			\textbf{\textit{dio 2. revizije}}\\
		
			 \textit{Potrebno je priložiti dijagram komponenti s pripadajućim opisom. Dijagram komponenti treba prikazivati strukturu cijele aplikacije.}