\chapter{Specifikacija programske potpore}
		
	\section{Funkcionalni zahtjevi}
			
			\textbf{\textit{dio 1. revizije}}\\
			
			\textit{Navesti \textbf{dionike} koji imaju \textbf{interes u ovom sustavu} ili  \textbf{su nositelji odgovornosti}. To su prije svega korisnici, ali i administratori sustava, naručitelji, razvojni tim.}\\
				
			\textit{Navesti \textbf{aktore} koji izravno \textbf{koriste} ili \textbf{komuniciraju sa sustavom}. Oni mogu imati inicijatorsku ulogu, tj. započinju određene procese u sustavu ili samo sudioničku ulogu, tj. obavljaju određeni posao. Za svakog aktora navesti funkcionalne zahtjeve koji se na njega odnose.}\\
			
			
			\noindent \textbf{Dionici:}
			
			\begin{packed_enum}
				
				\item Dionik 1
				\item Dionik 2				
				\item ...
				
			\end{packed_enum}
			
			\noindent \textbf{Aktori i njihovi funkcionalni zahtjevi:}
			
			
			\begin{packed_enum}
				\item  \underbar{Aktor 1 (inicijator) može:}
				
				\begin{packed_enum}
					
					\item funkcionalnost 1
					\item funkcionalnost 2
					\begin{packed_enum}
						
						\item  podfunkcionalnost 1 
						\item  podfunkcionalnost 2
				
					\end{packed_enum}
					\item  funkcionalnost 3
					
				\end{packed_enum}
			
				\item  \underbar{Aktor 2 (sudionik) može:}
				
				\begin{packed_enum}
					
					\item funkcionalnost 1
					\item funkcionalnost 2
					
				\end{packed_enum}
			\end{packed_enum}
			
			\eject 
			
			
				
			\subsection{Obrasci uporabe}
				
				\textbf{\textit{dio 1. revizije}}
				
				\subsubsection{Opis obrazaca uporabe}
					\textit{Funkcionalne zahtjeve razraditi u obliku obrazaca uporabe. Svaki obrazac je potrebno razraditi prema donjem predlošku. Ukoliko u nekom koraku može doći do odstupanja, potrebno je to odstupanje opisati i po mogućnosti ponuditi rješenje kojim bi se tijek obrasca vratio na osnovni tijek.}\\
					

					\noindent \underbar{\textbf{UC1 - Pregled liste aktivnih zahtjeva}}
					\begin{packed_item}
	
						\item \textbf{Glavni sudionik: }Korisnik
						\item  \textbf{Cilj:} Pregledati aktivne zahtjeve za pomoć
						\item  \textbf{Sudionici:} Baza podataka
						\item  \textbf{Preduvjet:} Korisnik je prijavljen u sustav
						\item  \textbf{Opis osnovnog tijeka:}
						
						\item[] \begin{packed_enum}
	
							\item Prikaz liste aktivnih zahtjeva udaljenih od korisnikove adrese za zadani radius 
							\item Korisnik može odabrati alternativni radius
							\item Odabir pojedinog zahtjeva
							\item Prikaz detalja o odabranom zahtjevu
						\end{packed_enum}
						
						\item  \textbf{Opis mogućih odstupanja:}
						
						\item[] \begin{packed_item}
	
							\item[1.a] Zahtjev nema unesenu lokaciju
							\item[] \begin{packed_enum}
								
								\item Zahtjevi bez lokacije prikazuju se svim korisnicima
							
							\end{packed_enum}
							
							
						\end{packed_item}
					\end{packed_item}
				
				\noindent \underbar{\textbf{UC1.1 - Javljanje na zahtjev}}
				\begin{packed_item}
					
					\item \textbf{Glavni sudionik: }Korisnik
					\item  \textbf{Cilj:} Inicijalizirati komunikaciju s korisnikom autorom
					\item  \textbf{Sudionici:} Korisnik autor zahtjeva, Baza podataka
					\item  \textbf{Preduvjet:} Oglas je aktivan
					\item  \textbf{Opis osnovnog tijeka:}
					
					\item[] \begin{packed_enum}
						
						\item Korisnik odabire zahtjev
						\item Korisnik se uvodi u bazu podataka kao potencijalni izvršitelj za odabrani zahtjev
						\item Korisniku autoru dolazi obavijest o novom potencijalnom izvršitelju
					\end{packed_enum}
					
					\item  \textbf{Opis mogućih odstupanja:}
					
					\item[] \begin{packed_item}
						
						\item[1.a] Korisnik autor može biti blokiran
						\item[] \begin{packed_enum}
							
							\item Zahtjevi blokiranih korisnika ne prikazuju se na glavnoj stranici zahtjeva
							\item Zahtjevi blokiranih korisnika vidljivi na njegovom profilu ne mogu biti odabrani za izvršavanje
							
						\end{packed_enum}
						
						
					\end{packed_item}
				\end{packed_item}
				
				
				\noindent \underbar{\textbf{UC1.2 - Filtriranje zahtjeva}}
				\begin{packed_item}
					
					\item \textbf{Glavni sudionik: }Korisnik
					\item  \textbf{Cilj:} Primjeniti različite filtre na zahtjeve
					\item  \textbf{Sudionici:}-
					\item  \textbf{Preduvjet:} Korisnik je prijavljen
					\item  \textbf{Opis osnovnog tijeka:}
					
					\item[] \begin{packed_enum}
						
						\item Korisnik otvara odabir filtera
						\item Korisnik označava željene filtre
						\item Potvrđivanje označenih filtera
						\item Prikaz filtriranih zahtjeva 
						\item Korisnik može poništiti sve primjenjene filtre
					\end{packed_enum}
					
					\item  \textbf{Opis mogućih odstupanja:}
					
					\item[] \begin{packed_item}
						
						\item[4.a] Nema zahtjeva za prikaz
						\item[] \begin{packed_enum}
							
							\item Prikazuje se odgovarajuća poruka
							
						\end{packed_enum}
						
						
					\end{packed_item}
				\end{packed_item}
			
			\noindent \underbar{\textbf{UC1.3 - Promjena lokacije izvršenja}}
			\begin{packed_item}
				
				\item \textbf{Glavni sudionik: }Korisnik
				\item  \textbf{Cilj:}Jednostavna izmjena korisnikove lokacije
				\item  \textbf{Sudionici:} Baza podataka
				\item  \textbf{Preduvjet:} Korisnik je prijavljen
				\item  \textbf{Opis osnovnog tijeka:}
				
				\item[] \begin{packed_enum}
					
					\item Korisnik odabire izmjenu lokacije
					\item Korisnik unosi novu lokaciju tekstom ili odabirom na karti
					\item Nova lokacija pohranjuje se u bazu
					\item Prikaz zahtjeva u odnosu na novu lokaciju
				\end{packed_enum}
				
				\item  \textbf{Opis mogućih odstupanja:}
				
				\item[] \begin{packed_item}
					
					\item[2.a] Unos nevaljane lokacije
					\item[] \begin{packed_enum}
						
						\item Unos lokacije u bazu se stornira
						\item Korisniku se prikazuje odgovarajuća poruka
						
					\end{packed_enum}
				
					
				\end{packed_item}
			\end{packed_item}
			
			
	
			\noindent \underbar{\textbf{UC2 - Registracija}}
			\begin{packed_item}
				
				\item \textbf{Glavni sudionik: }Javni korisnik
				\item  \textbf{Cilj:} Stvoriti račun u aplikaciji
				\item  \textbf{Sudionici:} Baza podataka
				\item  \textbf{Preduvjet:} -
				\item  \textbf{Opis osnovnog tijeka:}
				
				\item[] \begin{packed_enum}
					
					\item Korisnik unosi potrebne podatke
					\item Potvrda unesenih podataka
					\item Upis podataka u bazu
					\item Nakon uspješne registracije korisnik se preusmjerava na stranicu zahtjeva
				\end{packed_enum}
				
				\item  \textbf{Opis mogućih odstupanja:}
				
				\item[] \begin{packed_item}
					
					\item[1.a] Korisnik unosi već zauzeto korisničko ime i/ili e-mail
					\item[] \begin{packed_enum}
						
						\item Prikaz odgovarajuće poruke
						\item Omogućavanje ponovnog unosa neodgovarajućih podataka
						
					\end{packed_enum}
					\item[1.b] Nisu popunjena sva obavezna polja
					\item[] \begin{packed_enum}
						
						\item Prikaz odgovarajuće poruke
						\item Omogućavanje ponovnog unosa podataka
						
					\end{packed_enum}
					\item[4.a] Mogućnost odustajanja od registracije klikom na gumb
					
				\end{packed_item}
			\end{packed_item}
		
			\noindent \underbar{\textbf{UC3 - Zadavanje novog zahtjeva}}
			\begin{packed_item}
				
				\item \textbf{Glavni sudionik: }Korisnik
				\item  \textbf{Cilj:} Unijeti i opisati svoj zahtjev za pomoć
				\item  \textbf{Sudionici:} Baza podataka
				\item  \textbf{Preduvjet:} Korisnik je prijavljen
				\item  \textbf{Opis osnovnog tijeka:}
				
				\item[] \begin{packed_enum}
					
					\item Korisnik otvara komponentu za unos zahtjeva
					\item Unos opisa zahtjeva
					\item Unos vremena isteka
					\item Potvrda zahtjeva
					\item Unos zahtjeva u bazu podataka
				\end{packed_enum}
				
				\item  \textbf{Opis mogućih odstupanja:}
				
				\item[] \begin{packed_item}
					
					\item[2.a] Unos zahtjeva sa praznim opisom
					\item[] \begin{packed_enum}
						
						\item Prikaz poruke o minimalnoj duljini opisa od dva znaka
						
					\end{packed_enum}
					
				\end{packed_item}
			\end{packed_item}
		
		
			\noindent \underbar{\textbf{UC3.1 - Odabir lokacije zahtjeva}}
			\begin{packed_item}
				
				\item \textbf{Glavni sudionik: }Korisnik
				\item  \textbf{Cilj:} Opcionalan odabir lokacije zahtjeva
				\item  \textbf{Sudionici:} Baza podataka
				\item  \textbf{Preduvjet:} U tijeku je zadavanje novog zahtjeva
				\item  \textbf{Opis osnovnog tijeka:}
				
				\item[] \begin{packed_enum}
					
					\item Korisnik se odlučuje za postavljanje lokacije
					\item Odabir ručnog unosa ili unosa na karti
					\item Otvaranje polja ili karte za unos lokacije
					\item Potvrda lokacije
					\item Nastavak zadavanja zahtjeva
				\end{packed_enum}
				
				\item  \textbf{Opis mogućih odstupanja:}
				
				\item[] \begin{packed_item}
					
					\item[3.a] Unos prazne lokacije
					\item[] \begin{packed_enum}
						
						\item Zahtjevi bez lokacije vode se kao virtualni i prikazuju se svim korisnicima
						\item Virtualni zahtjevi polaze od pretpostavke da je lokacija irelevantna za uspješno izvršavanje
						
					\end{packed_enum}
					
					
				\end{packed_item}
			\end{packed_item}
		
		
			\noindent \underbar{\textbf{UC4 - Pregled vlastitog profila}}
			\begin{packed_item}
				
				\item \textbf{Glavni sudionik: }Korisnik
				\item  \textbf{Cilj:} Pregled vlasititih informacija i zahtjeva na profilu
				\item  \textbf{Sudionici:} -
				\item  \textbf{Preduvjet:} Korisnik je prijavljen
				\item  \textbf{Opis osnovnog tijeka:}
				
				\item[] \begin{packed_enum}
					
					\item Korisnik odabire opciju za prikaz profila
					\item Korisnik pregledava vlastite aktivne i izvršene zahtjeve
					\item Pregled vlastitih podataka korisničkog računa
				\end{packed_enum}
				
				
			\end{packed_item}
		
		
			\noindent \underbar{\textbf{UC4.1 - Upravljanje korisničkim podacima}}
			\begin{packed_item}
				
				\item \textbf{Glavni sudionik: }Korisnik
				\item  \textbf{Cilj:} Aktualizacija i izmjena 
				\item  \textbf{Sudionici:} Baza podataka
				\item  \textbf{Preduvjet:} Korisnik je prijavljen
				\item  \textbf{Opis osnovnog tijeka:}
				
				\item[] \begin{packed_enum}
					
					\item Korisnik odabire opciju izmjene podataka korisničkog računa
					\item Unos novih podataka
					\item Potvrda izmjena
					\item Spremanje izmjena u bazu podataka
				\end{packed_enum}
				
				\item  \textbf{Opis mogućih odstupanja:}
				
				\item[] \begin{packed_item}
					
					\item[3.a] Unos podataka u krivom formatu ili neispunjenje obaveznih polja
					\item[] \begin{packed_enum}
						
						\item Prikaz odgovarajuće poruke
						\item Ponovna mogućnost unosa podataka
						
					\end{packed_enum}
					
				\end{packed_item}
			\end{packed_item}
		
		
			\noindent \underbar{\textbf{UC4.3 - Brisanje korisničkog računa}}
			\begin{packed_item}
				
				\item \textbf{Glavni sudionik: }Korisnik
				\item  \textbf{Cilj:} Brisanje korisničkog računa i pratećih podataka 
				\item  \textbf{Sudionici:} Baza podataka
				\item  \textbf{Preduvjet:} -
				\item  \textbf{Opis osnovnog tijeka:}
				
				\item[] \begin{packed_enum}
					
					\item Odabir opcije brisanja korisničkog računa
					\item Korisnik mora dvostruko potvrditi akciju brisanja
					\item Korisnik u svakom trenutku prije finalne potvrde može odustati od brisanja
					\item Brisanje korisnika i popratnih podataka iz baze
				\end{packed_enum}
				
				
			\end{packed_item}
		
	
			\noindent \underbar{\textbf{UC5 - Pregled statistike}}
			\begin{packed_item}
				
				\item \textbf{Glavni sudionik: }Korisnik
				\item  \textbf{Cilj:} Prikaz statistika o korisnicima 
				\item  \textbf{Sudionici:} Baza podataka
				\item  \textbf{Preduvjet:} -
				\item  \textbf{Opis osnovnog tijeka:}
				
				\item[] \begin{packed_enum}
					
					\item Korisnik odabire opciju prikaza statistika
					\item Prikaz statistike u aplikaciji
					\item Povratak na prethodnu stranicu
			
				\end{packed_enum}
				
				
				
			\end{packed_item}
		
			\noindent \underbar{\textbf{UC6 - Pregled potencijalnih izvršitelja}}
			\begin{packed_item}
				
				\item \textbf{Glavni sudionik: }Korisnik
				\item  \textbf{Cilj:} Prikazati korisnike koji su se javili na pojedini zahtjev 
				\item  \textbf{Sudionici:} Baza podataka
				\item  \textbf{Preduvjet:} Postoje vlastiti aktivni zahtjevi
				\item  \textbf{Opis osnovnog tijeka:}
				
				\item[] \begin{packed_enum}
					
					\item Korisnik na zahtjevu bira opciju za prikaz potencijalnih izvršitelja
					\item Prikaz potencijalnih izvršitelja s opcijama za prihvaćanje i odbijanja
				\end{packed_enum}
				
				\item  \textbf{Opis mogućih odstupanja:}
				
				\item[] \begin{packed_item}
					
					\item[2.a] Odabrani zahtjev nema potencijalnih izvršitelja
					\item[] \begin{packed_enum}
						
						\item Prikaz poruke o nepostojanju potencijalnih izvršitelja
						
					\end{packed_enum}
					
				\end{packed_item}
			\end{packed_item}
		
			\noindent \underbar{\textbf{UC6.1 - Prihvaćanje javljanja na zahtjev}}
			\begin{packed_item}
				
				\item \textbf{Glavni sudionik: }Korisnik
				\item  \textbf{Cilj:} Prihvaćanje javljanja na zahtjev za pomoć 
				\item  \textbf{Sudionici:} Korisnik izvršitelj
				\item  \textbf{Preduvjet:} Postoji barem jedan potencijalni izvršitelj za zahtjev, Zahtjev je aktivan
				\item  \textbf{Opis osnovnog tijeka:}
				
				\item[] \begin{packed_enum}
					
					\item Prikaz liste potencijalnih izvršitelja
					\item Korisnik prihvaća pojedino javljanje na zahtjev
					\item Slanje obavijesti prihvaćenom korisniku
					\item Odbijanje ostalih potencijalnih izvršitelja
					\item Slanje obavijesti odbijenim korisnicima
					\item Pražnjenje liste potencijalnih izvršitelja
					\item Dodavanje prihvaćenog korisnika kao izvršitelja zahtjeva
				\end{packed_enum}
				
			\end{packed_item}
		
			\noindent \underbar{\textbf{UC6.2 - Odbijanje javljanja na zahtjev}}
			\begin{packed_item}
				
				\item \textbf{Glavni sudionik: }Korisnik
				\item  \textbf{Cilj:} Odbijanje pojedinog ili više potencijalnih izvršitelja
				\item  \textbf{Sudionici:} Korisnik izvršitelj
				\item  \textbf{Preduvjet:} Postoji barem jedan potencijalni izvršitelj za zahtjev
				\item  \textbf{Opis osnovnog tijeka:}
				
				\item[] \begin{packed_enum}
					
					\item Prikaz liste potencijalnih izvršitelja
					\item Korisnik odbija pojedino javljanje na zahtjev
					\item Odbijenom korisniku šalje se obavijest o odbijanju
				\end{packed_enum}
				
			\end{packed_item}
		
			\noindent \underbar{\textbf{UC7 - Administriranje korisnika}}
			\begin{packed_item}
				
				\item \textbf{Glavni sudionik: }Administrator
				\item  \textbf{Cilj:} Omogućiti rukovanje korisnicima 
				\item  \textbf{Sudionici:} Korisnici, Baza podataka
				\item  \textbf{Preduvjet:} -
				\item  \textbf{Opis osnovnog tijeka:}
				
				\item[] \begin{packed_enum}
					
					\item Na profilu korisnika administrator odabire opciju administriranja korisnika
					\item Administrator bira opciju privremenog blokiranja ili brisanja korisnika
					\item Administrator potvrđuje odabir
					\item Unos blokiranja/brisanja u bazu podataka
				\end{packed_enum}
				
			\end{packed_item}
		
		
			\noindent \underbar{\textbf{UC8 - Dodavanje novog administratora}}
			\begin{packed_item}
				
				\item \textbf{Glavni sudionik: }Administrator
				\item  \textbf{Cilj:} Postaviti nekog korisnika kao administratora 
				\item  \textbf{Sudionici:} Baza Podataka
				\item  \textbf{Preduvjet:} -
				\item  \textbf{Opis osnovnog tijeka:}
				
				\item[] \begin{packed_enum}
					
					\item Administrator na profilu korisnika odabire opciju postavljanja administratorskih ovlasti
					\item Administrator potvrđuje odabir
					
				\end{packed_enum}
				
				\item  \textbf{Opis mogućih odstupanja:}
				
				\item[] \begin{packed_item}
					
					\item[1.a] Pregledavanje profila korisnika koji već ima dodjeljene administratorske ovlasti
					\item[] \begin{packed_enum}
						
						\item Administratoru se ne omogućava ponovo postavljanje ovlasti
						
					\end{packed_enum}
					
				\end{packed_item}
			\end{packed_item}
		
			\noindent \underbar{\textbf{UC9 - Administriranje zahtjeva}}
			\begin{packed_item}
				
				\item \textbf{Glavni sudionik: }Administrator
				\item  \textbf{Cilj:} Brisanje neprihvatljivih zahtjeva 
				\item  \textbf{Sudionici:} Baza podataka
				\item  \textbf{Preduvjet:} -
				\item  \textbf{Opis osnovnog tijeka:}
				
				\item[] \begin{packed_enum}
					
					\item Administrator odabire opciju brisanja zahtjeva
					\item Administrator potvrđuje odabir
					\item Autoru zahtjeva dolazi obavijest o brisanju zahtjeva
					\item Potencijalnim izvršiteljima se šalje obavijest o brisanju zahtjeva
				\end{packed_enum}
				
				\item  \textbf{Opis mogućih odstupanja:}
				
				\item[] \begin{packed_item}
					
					\item[2.a] Javljanje na zahtjev je već prihvaćeno i izvršitelj je postavljen
					\item[] \begin{packed_enum}
						
						\item Izvršitelj dobiva obavijest o brisanju zahtjeva
						
					\end{packed_enum}
					
				\end{packed_item}
			\end{packed_item}
		
		
			\noindent \underbar{\textbf{UC10 - Pregled profila drugih korisnika}}
			\begin{packed_item}
				
				\item \textbf{Glavni sudionik: }Korisnik
				\item  \textbf{Cilj:} Pregled profila korisnika 
				\item  \textbf{Sudionici:} -
				\item  \textbf{Preduvjet:} -
				\item  \textbf{Opis osnovnog tijeka:}
				
				\item[] \begin{packed_enum}
					
					\item Prikaz osnovnih podataka o korisniku, njegovih zahtjeva i zahtjeva koje je on izvršio
					\item Korisnik može pregledavati zahtjeve profila
					\item Korisnik može pregledati lanac povjerenja, komentare i ocjenu profila korisnika
				\end{packed_enum}
				
			\end{packed_item}
		
		
			\noindent \underbar{\textbf{UC11 - Ocjenjivanje korisnika}}
			\begin{packed_item}
				
				\item \textbf{Glavni sudionik: }Korisnik
				\item  \textbf{Cilj:} Ocjena korisnika i/ili ocjena izvršenja zahtjeva uz popratan komentar 
				\item  \textbf{Sudionici:} Baza podataka
				\item  \textbf{Preduvjet:} -
				\item  \textbf{Opis osnovnog tijeka:}
				
				\item[] \begin{packed_enum}
					
					\item Korisnik inicira ocjenjivanje na profilu ili se ocjenjivanje pokreće nakon izvršenja zahtjeva
					\item Korisnik izabire ocjenu od 1 do 5
					\item Korisnik opcionalno unosi komentar
					\item Korisnik potvrđuje svoj odabir
					\item Ocjena i komentar se
				\end{packed_enum}
				
				\item  \textbf{Opis mogućih odstupanja:}
				
				\item[] \begin{packed_item}
					
					\item[2.a] Ocjena se može, ali ne mora odnositi na izvršavanje specifičnog zahtjeva
					\item[] \begin{packed_enum}
						
						\item Ukoliko se ocjena odnosi na izvršavanje zahtjeva, u bazu se upisuje o kojem se zahtjevu radi
						
					\end{packed_enum}
					
				\end{packed_item}
			\end{packed_item}
		
		
			\noindent \underbar{\textbf{UC12 - Izvršavanje zahtjeva}}
			\begin{packed_item}
				
				\item \textbf{Glavni sudionik: }Korisnik
				\item  \textbf{Cilj:} Označavanje zahtjeva izvršenim 
				\item  \textbf{Sudionici:} Baza podataka
				\item  \textbf{Preduvjet:} Korisnik je postavljen kao izvršitelj zahtjeva
				\item  \textbf{Opis osnovnog tijeka:}
				
				\item[] \begin{packed_enum}
					
					\item Korisnik odabire zahtjev koji izvršava
					\item Korisnik označuje zahtjev izvršenim
					\item Korisniku autoru šalje se obavijest o izvršenju
					\item Inicira se ocjenjivanje korisnika

				\end{packed_enum}
				
			\end{packed_item}
		
		
			\noindent \underbar{\textbf{UC4.1 - Upravljanje korisničkim podacima}}
			\begin{packed_item}
				
				\item \textbf{Glavni sudionik: }Korisnik
				\item  \textbf{Cilj:} Aktualizacija i izmjena 
				\item  \textbf{Sudionici:} $<$sudionici$>$
				\item  \textbf{Preduvjet:} $<$preduvjet$>$
				\item  \textbf{Opis osnovnog tijeka:}
				
				\item[] \begin{packed_enum}
					
					\item $<$opis korak jedan$>$
					\item $<$opis korak dva$>$
					\item $<$opis korak tri$>$
					\item $<$opis korak četiri$>$
					\item $<$opis korak pet$>$
				\end{packed_enum}
				
				\item  \textbf{Opis mogućih odstupanja:}
				
				\item[] \begin{packed_item}
					
					\item[2.a] $<$opis mogućeg scenarija odstupanja u koraku 2$>$
					\item[] \begin{packed_enum}
						
						\item $<$opis rješenja mogućeg scenarija korak 1$>$
						\item $<$opis rješenja mogućeg scenarija korak 2$>$
						
					\end{packed_enum}
					\item[2.b] $<$opis mogućeg scenarija odstupanja u koraku 2$>$
					\item[3.a] $<$opis mogućeg scenarija odstupanja  u koraku 3$>$
					
				\end{packed_item}
			\end{packed_item}
		
		
			\noindent \underbar{\textbf{UC4.1 - Upravljanje korisničkim podacima}}
			\begin{packed_item}
				
				\item \textbf{Glavni sudionik: }Korisnik
				\item  \textbf{Cilj:} Aktualizacija i izmjena 
				\item  \textbf{Sudionici:} $<$sudionici$>$
				\item  \textbf{Preduvjet:} $<$preduvjet$>$
				\item  \textbf{Opis osnovnog tijeka:}
				
				\item[] \begin{packed_enum}
					
					\item $<$opis korak jedan$>$
					\item $<$opis korak dva$>$
					\item $<$opis korak tri$>$
					\item $<$opis korak četiri$>$
					\item $<$opis korak pet$>$
				\end{packed_enum}
				
				\item  \textbf{Opis mogućih odstupanja:}
				
				\item[] \begin{packed_item}
					
					\item[2.a] $<$opis mogućeg scenarija odstupanja u koraku 2$>$
					\item[] \begin{packed_enum}
						
						\item $<$opis rješenja mogućeg scenarija korak 1$>$
						\item $<$opis rješenja mogućeg scenarija korak 2$>$
						
					\end{packed_enum}
					\item[2.b] $<$opis mogućeg scenarija odstupanja u koraku 2$>$
					\item[3.a] $<$opis mogućeg scenarija odstupanja  u koraku 3$>$
					
				\end{packed_item}
			\end{packed_item}
		
			
			
					
				\subsubsection{Dijagrami obrazaca uporabe}
					
					\textit{Prikazati odnos aktora i obrazaca uporabe odgovarajućim UML dijagramom. Nije nužno nacrtati sve na jednom dijagramu. Modelirati po razinama apstrakcije i skupovima srodnih funkcionalnosti.}
				\eject		
				
			\subsection{Sekvencijski dijagrami}
				
				\textbf{\textit{dio 1. revizije}}\\
				
				\textit{Nacrtati sekvencijske dijagrame koji modeliraju najvažnije dijelove sustava (max. 4 dijagrama). Ukoliko postoji nedoumica oko odabira, razjasniti s asistentom. Uz svaki dijagram napisati detaljni opis dijagrama.}
				\eject
	
		\section{Ostali zahtjevi}
		
		\begin{itemize}
			\item Aplikacija treba biti izvedena kao web aplikacija prilagođena mobilnom uređaju.
			\item Sustav mora podržavati rad više korisnika u stvarnom vremenu.
			\item Sustav kao valutu koristi HRK.
			\item Procesiranje bilo kakve korisničke interakcije sa sustavom ne bi trebalo trajati duže od par sekundi.
			\item Administratori su dodijeljeni po geografskim lokacijama.
			\item Sustav mora podržavati hrvatske dijakritičke znakove.
			\item Informacije o zahtjevima moraju biti redovno ažurirane.
			\item U sustavu je potrebno registrirati barem 5 korisnika te 2 administratora.
			\item Korisničko sučelje treba biti jednostavno za korištenje.
		\end{itemize}
			 
			 
			 
	