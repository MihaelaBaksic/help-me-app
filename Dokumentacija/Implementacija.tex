\chapter{Implementacija i korisničko sučelje}
		
		
		\section{Korištene tehnologije i alati}

		Komunikacija našem u timu je realizirana korištenjem aplikacija Discord\footnote{https://discord.com/} i Whatsapp\footnote{https://www.whatsapp.com/}.\\ Za izradu UML dijagrama korišten je alat Astah Professional\footnote{http://astah.net/editions/professional}, a kao sustav za upravljanje
izvornim kodom Git\footnote{https://git-scm.com/}. Udaljeni repozitorij projekta je dostupan na web platformi GitLab\footnote{https://gitlab.com/}, a kao razvojno okruženje korišten je IntelliJ IDEA\footnote{https://www.jetbrains.com/idea/} - tvrtke JetBrains\footnote{https://www.jetbrains.com/} te Microsoftov Visual Studio Code\footnote{https://visualstudio.microsoft.com/
}. 
	\\
	 Naša aplikacija napisana je koristeći Spring Framework\footnote{https://spring.io/} i jezik Java\footnote{https://www.oracle.com/java/} za izradu backenda te React\footnote{https://reactjs.org/} - JavaScript library za izradu frontenda. React je open-source, front end, JavaScript knjižnica za izgradnju korisničkog sučelja ili UI komponenti. Održavaju ga Facebook i zajednica pojedinačnih programera i tvrtki. Spring razvojno okruženje je Java platforma koja pruža široki panel opcija kao podrški
razvoju Java aplikacija. Spring rukuje infrastrukturom, tako da programer može usmjeriti svoj
fokus na razvoj aplikacije. \\
Tijekom izrađivanja aplikacije koristili smo i Postman\footnote{https://www.postman.com/} za simuliranje HTTP zahtjeva nad aplikacijom, Google Drive\footnote{https://workspace.google.com/products/drive/} za efikasno dijeljenje organizacijskih informacija i raznih datoteka. 
Baza podataka se nalazi na poslužitelju u oblaku Amazon
Web Services\footnote{https://aws.amazon.com/}.			
			
			\eject 
		
	
		\section{Ispitivanje programskog rješenja}
			
			\textbf{\textit{dio 2. revizije}}\\
			
			 \textit{U ovom poglavlju je potrebno opisati provedbu ispitivanja implementiranih funkcionalnosti na razini komponenti i na razini cijelog sustava s prikazom odabranih ispitnih slučajeva. Studenti trebaju ispitati temeljnu funkcionalnost i rubne uvjete.}
	
			
			\subsection{Ispitivanje komponenti}
			\textit{Potrebno je provesti ispitivanje jedinica (engl. unit testing) nad razredima koji implementiraju temeljne funkcionalnosti. Razraditi \textbf{minimalno 6 ispitnih slučajeva} u kojima će se ispitati redovni slučajevi, rubni uvjeti te izazivanje pogreške (engl. exception throwing). Poželjno je stvoriti i ispitni slučaj koji koristi funkcionalnosti koje nisu implementirane. Potrebno je priložiti izvorni kôd svih ispitnih slučajeva te prikaz rezultata izvođenja ispita u razvojnom okruženju (prolaz/pad ispita). }
			
			
			
			\subsection{Ispitivanje sustava}
			
			 \textit{Potrebno je provesti i opisati ispitivanje sustava koristeći radni okvir Selenium\footnote{\url{https://www.seleniumhq.org/}}. Razraditi \textbf{minimalno 4 ispitna slučaja} u kojima će se ispitati redovni slučajevi, rubni uvjeti te poziv funkcionalnosti koja nije implementirana/izaziva pogrešku kako bi se vidjelo na koji način sustav reagira kada nešto nije u potpunosti ostvareno. Ispitni slučaj se treba sastojati od ulaza (npr. korisničko ime i lozinka), očekivanog izlaza ili rezultata, koraka ispitivanja i dobivenog izlaza ili rezultata.\\ }
			 
			 \textit{Izradu ispitnih slučajeva pomoću radnog okvira Selenium moguće je provesti pomoću jednog od sljedeća dva alata:}
			 \begin{itemize}
			 	\item \textit{dodatak za preglednik \textbf{Selenium IDE} - snimanje korisnikovih akcija radi automatskog ponavljanja ispita	}
			 	\item \textit{\textbf{Selenium WebDriver} - podrška za pisanje ispita u jezicima Java, C\#, PHP koristeći posebno programsko sučelje.}
			 \end{itemize}
		 	\textit{Detalji o korištenju alata Selenium bit će prikazani na posebnom predavanju tijekom semestra.}
			
			\eject 
		
		
		\section{Dijagram razmještaja}
			
			Na slici 5.1 nalazi se dijagram razmještaja koji prikazuje raspored programske podrške aplikaciji unutar sklopovlja.
			
			Sa strane korisnika aplikacije, sklopovlje predstavlja njegovo računalo, mobilni telefon ili bilo koji drugi uređaj s pristupom internetu i instaliranim browserom.
			Internetski browser zadužen je za uspostavu konekcije sa poslužiteljem koja će omogućiti slanje i posluživanje zahtjeva.
			Sa strane poslužitelja sklopovlje predstavlja serversko računalo.
			Dvije glavne usluge koje se nalaze na serverskom računalu su baza podataka i web server koje su nužne za posluživanje zahtjeva.
			
			Komunikacija između korisnika i poslužitelja vrši se putem \textit{stateless} HTTP protokola. 
			Tijekom slanja zahtjeva serveru, korisnikov browser ima otvorenu jednu HTTP vezu prema serveru.
			S druge strane server može posluživati više zahtjeva iz više različitih izvora te stoga može imati i više aktivnih HTTP veza.
			
			\begin{figure}[h]
				\includegraphics[height=0.5\textheight]{dijagramRazmjestaja}
				\caption{Dijagram razmještaja}
			\end{figure} 
			
			\eject 
		
		\section{Upute za puštanje u pogon}
		
			\textbf{\textit{dio 2. revizije}}\\
		
			 \textit{U ovom poglavlju potrebno je dati upute za puštanje u pogon (engl. deployment) ostvarene aplikacije. Na primjer, za web aplikacije, opisati postupak kojim se od izvornog kôda dolazi do potpuno postavljene baze podataka i poslužitelja koji odgovara na upite korisnika. Za mobilnu aplikaciju, postupak kojim se aplikacija izgradi, te postavi na neku od trgovina. Za stolnu (engl. desktop) aplikaciju, postupak kojim se aplikacija instalira na računalo. Ukoliko mobilne i stolne aplikacije komuniciraju s poslužiteljem i/ili bazom podataka, opisati i postupak njihovog postavljanja. Pri izradi uputa preporučuje se \textbf{naglasiti korake instalacije uporabom natuknica} te koristiti što je više moguće \textbf{slike ekrana} (engl. screenshots) kako bi upute bile jasne i jednostavne za slijediti.}
			
			
			 \textit{Dovršenu aplikaciju potrebno je pokrenuti na javno dostupnom poslužitelju. Studentima se preporuča korištenje neke od sljedećih besplatnih usluga: \href{https://aws.amazon.com/}{Amazon AWS}, \href{https://azure.microsoft.com/en-us/}{Microsoft Azure} ili \href{https://www.heroku.com/}{Heroku}. Mobilne aplikacije trebaju biti objavljene na F-Droid, Google Play ili Amazon App trgovini.}
			
			
			\eject 