\chapter{Zaključak i budući rad}

		
		 
		 \section{Izrada projekta i tehnički izazovi}
		 
		 Izrada projekta protezala se kroz period od 12 tjedana. Prvi dio izrade fokusirao se na definiranje korisničkih zahtjeva i zahtjeva sustava, dok se drugi dio okrenuo ka implementaciji definiranog sustava.
		 
		 Imajući na umu da nitko iz projektne grupe nije bio upoznat sa tehnologijama koje su se koristile (Spring Framework i React), glavni izazov predstavljalo je upoznavanje s novim tehnologijama. \newline
		 Inicijalno, to je otežavalo vremensko planiranje i nošenje sa klasičnim problemima uporabe Springa i Reacta koji se mogu zaobići korištenjem koncepta \textit{best practice}. \newline 
		 
		 \subsection{Tehnički izazovi korištenja radnog okvira Spring}
		 Najveći izazov prilikom korištenja radnog okvira Spring bili su u uspostavi autentikacije i autorizacije. Radi jednostavnosti izvedbe inicijalne inačice, odlučeno je za konfiguraciju sigurnosti aplikacije koristiti već postojeću Springovu klasu \textit{WebSecurityConfigurerAdapter}. Najduže je trajalo uspostavljanje konfiguracije autentikacije putem \textit{UserDetailsService}-a iz razloga nepotpunosti i manjkavosti informacija o funkcionalnostima navedenog service-a na internetu i čestih neočekivanih ponašanja koja su posljedica različitih konfiguracija sigurnosti u referentnim primjerima u našoj aplikaciji. Autentikacija je uspješno uspostavljena pregledavanjem implementacije korištenih klasa i službene dokumentacije. 
		 Manjkavost koju uviđamo u našoj izvedbi sigurnosti je nekorištenje tokena, već svaki zahtjev mora sadržavati autentikacijske podatke \textit{username} i \textit{password}.
			 
	
		
		\subsection{Izazovi korištenja knjižnice React}
		Glavnina tehničkih izazova korištenja knjižnice React je proizašla iz potrebe za adaptacijom komponenti koje React nudi. Kao kompromis između pisanja vlastitih i relativno teške adaptacije već postojećih komponenti, glavni \textit{layout} web stranice izrađen je samostalno korištenjem CSS-a, dok su se za pojedine komponente unutar stranice koristile komponente \textit{Semantic-UI} i \textit{Bootstrap}. \newline
		
		
		 \section{Budući rad}
		 S obzirom na kratko vrijeme za izvedbu aplikacije i ograničeno prethodno iskustvo projektnog tima, aplikacija ostaje otvorena za neka poboljšanja. U ovom odjeljku opisani su prijedlozi za restrukturiranje dijela funkcionalnosti te koje dodatne funkcionalnosti bi poboljšale aplikaciju.\newline 
		 
		 Restrukturiranje sigurnosti i izmjene potrebne za uvođenje \textit{JWT tokena} kao sredstva autentikacije prva je i najbitnija potrebna izmjena. Time se osigurava veća sigurnost podataka u odnosu na slanje \textit{passworda} u svakom HTTP zahtjevu. Dodatno, e-mail verifikacijom unesene adrese prilikom registracije potrebno je osigurati da se nitko osim stvarnog posjednika adrese ne može registrirati u sustav. \newline
		  
		 
		 Kako bi se osigurala kvalitetnija komunikacija i interakcija između korisnika aplikacije te bolji \textit{user experience} dodatne implementacija funkcionalnosti pretraživanja korisnika i zahtjeva te \textit{chat} koje nisu izvedene u prvoj inačici je nužna. \newline
		 
		 Od zahtjevanih i predviđenih funkcionalnosti aplikacije nisu implementirani:
		 \begin{packed_enum}
		 	
		 	\item Brisanje korisničkog računa
		 	\item Prikaz lanaca povjerenja
		 	\item Čuvanje podataka na udaljenoj bazi podataka 

		\end{packed_enum}		
		Predviđene i neimplementirane funkcionalnosti bit će uključene u iduću inačicu aplikacije. 	
		 
		 \eject
		 
		 \section{Zaključak}
		 Izvođenje ovog projekta uvelike je pridonijelo praktičnom iskustvu svih članova projektne grupe. Najbitnije znanje koje je projektna grupa dobila je uvid u cjelokupni proces razvoja određenog programskog proizvoda, počevši sa planiranjem, razradom pa sve do \textit{deployment}-a. To znanje je učinilo dosadašnja znanja članova projektne grupe kompletnijima i primjenjivijima. 
		 
		 Dobra komunikacija članova projektnog tima pokazala se kao bitan faktor u uspješnom svladavanju prepreka. 
		 
		 Također, projekt je potaknuo grupu na aktivnije korištenje već gotovih knjižnica i komponenata i njihovu adaptaciju za specifične potrebe projekta.
		 \eject
		 
		 
		
		\eject 